\documentclass{article}
\usepackage{graphicx} % Required for inserting images
% Options for packages loaded elsewhere
\PassOptionsToPackage{unicode}{hyperref}
\PassOptionsToPackage{hyphens}{url}
%
\usepackage{amsmath,amssymb}
\usepackage{lmodern}
\usepackage{iftex}
\ifPDFTeX
  \usepackage[T1]{fontenc}
  \usepackage[utf8]{inputenc}
  \usepackage{textcomp} % provide euro and other symbols
\else % if luatex or xetex
  \usepackage{unicode-math}
  \defaultfontfeatures{Scale=MatchLowercase}
  \defaultfontfeatures[\rmfamily]{Ligatures=TeX,Scale=1}
\fi
% Use upquote if available, for straight quotes in verbatim environments
\IfFileExists{upquote.sty}{\usepackage{upquote}}{}
\IfFileExists{microtype.sty}{% use microtype if available
  \usepackage[]{microtype}
  \UseMicrotypeSet[protrusion]{basicmath} % disable protrusion for tt fonts
}{}
\makeatletter
\@ifundefined{KOMAClassName}{% if non-KOMA class
  \IfFileExists{parskip.sty}{%
    \usepackage{parskip}
  }{% else
    \setlength{\parindent}{0pt}
    \setlength{\parskip}{6pt plus 2pt minus 1pt}}
}{% if KOMA class
  \KOMAoptions{parskip=half}}
\makeatother
\usepackage{xcolor}
\usepackage{graphicx}
\makeatletter
\def\maxwidth{\ifdim\Gin@nat@width>\linewidth\linewidth\else\Gin@nat@width\fi}
\def\maxheight{\ifdim\Gin@nat@height>\textheight\textheight\else\Gin@nat@height\fi}
\makeatother
% Scale images if necessary, so that they will not overflow the page
% margins by default, and it is still possible to overwrite the defaults
% using explicit options in \includegraphics[width, height, ...]{}
\setkeys{Gin}{width=\maxwidth,height=\maxheight,keepaspectratio}
% Set default figure placement to htbp
\makeatletter
\def\fps@figure{htbp}
\makeatother
\setlength{\emergencystretch}{3em} % prevent overfull lines
\providecommand{\tightlist}{%
  \setlength{\itemsep}{0pt}\setlength{\parskip}{0pt}}
\setcounter{secnumdepth}{-\maxdimen} % remove section numbering
\ifLuaTeX
  \usepackage{selnolig}  % disable illegal ligatures
\fi
\IfFileExists{bookmark.sty}{\usepackage{bookmark}}{\usepackage{hyperref}}
\IfFileExists{xurl.sty}{\usepackage{xurl}}{} % add URL line breaks if available
\urlstyle{same} % disable monospaced font for URLs
\hypersetup{
  hidelinks,
  pdfcreator={LaTeX via pandoc}}

\author{}
\date{}

\title{Normal Distribution - Standard Normal Distribution}
\author{Rylei Mindrum}
\date{October 2024}

\begin{document}

\maketitle

1

Normal Distribution Standard Normal Distribution

\emph{•} There is a special normal distribution.

\emph{•} We call it the standard normal distribution.\\
\emph{•} The mean is \emph{µ} = 0.

\emph{•} Then standard deviation is \emph{$\sigma$} = 1.

\emph{•} We call the random variable \emph{Z}.

For this activity, you need the applet at

If you have issues you can instead use

\emph{•} The z value is for the standard normal distribution.\\
\emph{•} You can drag the dividing line to change the Z value. This line
divides the area under the normal curve into the area to the left and
the area to the right.

\emph{•} If you click the Both tails button, then you can see the area
in the middle.\emph{•} You can click Right tail, you can see the area to
the right.

\emph{•} You can also use the left and right arrows on your keyboard to
change Z.

\emph{•} Remember that for a continuous distribution, \emph{P} (\emph{Z >=}2) is the same as \emph{P} (\emph{Z \textgreater{}} 2).\\
\emph{•} You won\textquotesingle t be able to get the exact Z value you
want by dragging the line with the mouse. But your answers should be
close to mine.

1. Play with the applet for a few minutes. Click the buttons and drag
the dividing line until you feel comfortable

working with it.

2. Use the applet to nd the following probabilities. Draw a picture for
each problem.

(a) Find the area under the curve to the right of \emph{z} = 2. In
symbols this is \emph{P}(\emph{Z \textgreater{}} 2).
\begin{itemize}
    \item (1 - .9772) = .0228
\end{itemize}

(b) Find the area under the curve to the left of \emph{z} = 2. In
symbols this is \emph{P} (\emph{Z \textless{}} 2).
\begin{itemize}
    \item .9772
\end{itemize}
(c) What is the relationship between \emph{P}(\emph{Z \textgreater{}} 2)
and \emph{P} (\emph{Z \textless{}} 2)?
\begin{itemize}
    \item They're Compliments
    \item (1-.9772)=.0228
\end{itemize}
(d) Find the area under the curve to the right of \emph{z} = \emph{-}2.
In symbols this is \emph{P} (\emph{Z \textgreater{} -}2).
\begin{itemize}
    \item (1-.0228)=.9772
    \item .9772
\end{itemize}
2

\begin{quote}
(e) \emph{P} (\emph{Z >= -}2)
\begin{itemize}
    \item > and >= are the same
    \item .9772
\end{itemize}
(f) \emph{P} (\emph{Z \textless{}} 1)
\begin{itemize}
    \item .8413
\end{itemize}
(g) Find the area under the curve between \emph{z} = \emph{-}1 and
\emph{z} = 1. In symbols this is \emph{P} (\emph{-}1 \emph{\textless{} Z
\textless{}} 1).
\begin{itemize}
    \item .6827
\end{itemize}
(h) \emph{P} (\emph{-}1\emph{.}2 \emph{\textless{} Z \textless{}}
1\emph{.}2)
\end{quote}
\begin{itemize}
    \item .77
\end{itemize}
3. The empirical rule says that

\begin{quote}
\emph{•} approximately 68\% of the data is within 1 standard deviation
of the mean.

\emph{•} approximately 95\% of the data is within 2 standard deviations
of the mean.

\emph{•} approximately 99.7\% of the data is within 3 standard
deviations of the mean.

We can find the exact percentages for a normal distribution using the
applet.

(a) To find the percentage of data within 1 standard deviation of the
mean, we need to find \emph{P} (\emph{-}1 \emph{\textless{} Z \textless{}}
1). Draw a picture.
\begin{itemize}
    \item pictures on separate attachment
    \item .6827
\end{itemize}
(b) What is the exact percentage of data within 2 standard deviations of
the mean? Find \emph{P} (\emph{-}2 \emph{\textless{} Z \textless{}} 2).
Draw a picture.
\begin{itemize}
    \item pictures on separate attachment
    \item .9545
\end{itemize}
(c) What is the exact percentage of data within 3 standard deviations of
the mean? Find \emph{P} (\emph{-}3 \emph{\textless{} Z \textless{}} 3).
Draw a picture.
\end{quote}
\begin{itemize}
    \item pictures on separate attachment
    \item .9973
\end{itemize}
3

4. Sometimes we want to know the z value that divides the curve into
certain probabilities. Draw a picture for each problem.

\begin{quote}
(a) What \emph{z} value has 0.3 area to the left?
\begin{itemize}
    \item pictures on separate attachment
    \item -.521
\end{itemize}
(b) What \emph{z} value has 0.7 area to the right?
\begin{itemize}
    \item pictures on separate attachment
    \item -.5516
\end{itemize}
(c) What \emph{z} value has 0.3 area to the right?
\begin{itemize}
    \item pictures on separate attachment
    \item .5107
\end{itemize}
(d) Find the \emph{z} value such that \emph{P} (\emph{Z \textgreater{}
z}) = \emph{.}4.
\begin{itemize}
    \item pictures on separate attachment
    \item .4002
\end{itemize}
(This is really fancy notation to say finnd the \emph{z} value that has .4
probability to the right.

Some books say find the \emph{k} value such that \emph{P} (\emph{Z
\textgreater{} k}) = \emph{.}4.)

(e) What \emph{z} value has 0.1 area to the right?
\begin{itemize}
    \item pictures on separate attachment
    \item 1.282
\end{itemize}
(f) What \emph{z} value do we need so that the area in the middle
between \emph{z} and \emph{-z} is 0.5?
\begin{itemize}
    \item pictures on separate attachment
    \item .686
\end{itemize}
\end{quote}

5.1.1
\begin{itemize}
    \item 1.1 Suppose that Z ~ N (0, 1). Find: 
    \begin{itemize}
    \item[(a)] \( P(Z \geq 1.34) \) \\
    This is equivalent to \( 1 - P(Z < 1.34) \), where \( P(Z < 1.34) \) is the CDF at 1.34:
    \[
    P(Z < 1.34) \approx 0.0901 \quad \text{so} \quad P(Z \geq 1.34) = 1 - 0.0901 = 0.9099
    \]

    \item[(b)] \( P(Z \geq -0.22) \) \\
    This is equivalent to \( 1 - P(Z < -0.22) \), where \( P(Z < -0.22) \) is the CDF at -0.22:
    \[
    P(Z < -0.22) \approx 0.4129 \quad \text{so} \quad P(Z \geq -0.22) = 1 - 0.4129 = 0.5871
    \]

    \item[(c)] \( P(-2.19 \leq Z \leq 0.43) \) \\
    We calculate this as \( P(Z \leq 0.43) - P(Z \leq -2.19) \). From the standard normal table:
    \[
    P(Z \leq 0.43) \approx 0.6664, \quad P(Z \leq -2.19) \approx 0.0143
    \]
    \[
    P(-2.19 \leq Z \leq 0.43) = 0.6664 - 0.0143 = 0.6521
    \]

    \item[(d)] \( P(0.09 \leq Z \leq 1.76) \) \\
    We calculate this as \( P(Z \leq 1.76) - P(Z \leq 0.09) \):
    \[
    P(Z \leq 1.76) \approx 0.9608, \quad P(Z \leq 0.09) \approx 0.5359
    \]
    \[
    P(0.09 \leq Z \leq 1.76) = 0.9608 - 0.5359 = 0.4249
    \]

    \item[(e)] \( P(|Z| \leq 0.38) \) \\
    This is equivalent to \( P(-0.38 \leq Z \leq 0.38) \). We use the CDF values at \( Z = 0.38 \) and \( Z = -0.38 \) (which are equal):
    \[
    P(Z \leq 0.38) \approx 0.6480, \quad P(Z \leq -0.38) \approx 1 - 0.6480 = 0.3520
    \]
    \[
    P(|Z| \leq 0.38) = 0.6480 - 0.3520 = 0.2960
    \]

    \item[(f)] The value of \( x \) for which \( P(Z \leq x) = 0.55 \) \\
    From the standard normal table, the value of \( x \) corresponding to \( P(Z \leq x) = 0.55 \) is approximately:
    \[
    x \approx 0.13
    \]

    \item[(g)] The value of \( x \) for which \( P(Z \geq x) = 0.72 \) \\
    This implies that \( P(Z \leq x) = 0.28 \). From the standard normal table:
    \[
    x \approx -0.58
    \]

    \item[(h)] The value of \( x \) for which \( P(|Z| \leq x) = 0.31 \) \\
    This implies \( P(-x \leq Z \leq x) = 0.31 \), so we find the CDF value for \( x \) such that \( P(Z \leq x) = \frac{0.31 + 1}{2} = 0.655 \). From the standard normal table:
    \[
    x \approx 0.39
    \]
\end{itemize}


5.1.2 Suppose that Z ∼ N (0, 1). Find: 
    \item[(a)] \( P(Z \geq -0.77) \) \\
    This is equivalent to \( 1 - P(Z \leq -.77) \):
    \[    P(Z \leq -.77) \approx = .2206    \]

    \item[(b)] \( P(Z \geq 0.32) \) \\
    This is equivalent to \( 1 - P(Z \leq 0.32) \):
    \[    P(Z \leq 0.32) \approx 0.6255 \quad \text{so} \quad P(Z \geq 0.32) = 1 - 0.6255 = 0.3745    \]

    \item[(c)] \( P(-3.09 \leq Z \leq -1.59) \) \\
    We calculate this as \( P(Z \leq -1.59) - P(Z \leq -3.09) \):
    \[    P(Z \leq -1.59) \approx 0.0559, \quad P(Z \leq -3.09) \approx 0.0010    \]
    \[    P(-3.09 \leq Z \leq -1.59) = 0.0559 - 0.0010 = 0.0549    \]

    \item[(d)] \( P(-0.82 \leq Z \leq 1.80) \) \\
    We calculate this as \( P(Z \leq 1.80) - P(Z \leq -0.82) \):
    \[    P(Z \leq 1.80) \approx 0.9641, \quad P(Z \leq -0.82) \approx 0.2061    \]
    \[    P(-0.82 \leq Z \leq 1.80) = 0.9641 - 0.2061 = 0.7580    \]

    \item[(e)] \( P(|Z| \geq 0.91) \) \\
    This is equivalent to \( 2 \times P(Z \geq 0.91) \). From the standard normal table:
    \[    P(Z \geq 0.91) \approx 1 - 0.8186 = 0.1814    \]
    \[    P(|Z| \geq 0.91) = 2 \times 0.1814 = 0.3628    \]

    \item[(f)] The value of \( x \) for which \( P(Z \leq x) = 0.23 \) \\
    From the standard normal table, the value of \( x \) corresponding to \( P(Z \leq x) = 0.23 \) is approximately:
    \[    x \approx -0.74    \]

    \item[(g)] The value of \( x \) for which \( P(Z \geq x) = 0.51 \) \\
    This implies that \( P(Z \leq x) = 0.49 \). From the standard normal table:
    \[    x \approx -0.025    \]

    \item[(h)] The value of \( x \) for which \( P(|Z| \geq x) = 0.42 \) \\
    This implies that \( P(|Z| \leq x) = 0.58 \), so we find \( x \) such that \( P(Z \leq x) = 0.79 \). From the standard normal table:
    \[    x \approx 0.81    \]
\end{itemize}

\end{document}

\end{document}
